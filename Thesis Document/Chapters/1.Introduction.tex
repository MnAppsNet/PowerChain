\chapter{Introduction} \label{introduction}
As the consumer level energy generation solutions become cheaper and more available, new opportunities arise in the energy trading sector. Consumers can produce and trade energy on a 
local network and distribution systems operators (DSO) are utilizing distributed energy resources (DERs) such as rooftop solar photovoltaic units (PV) and wind generators to offer affordable and eco-friendly energy 
solutions.
\section{Background}
In the past, electricity distribution primarily adhered to a centralized architecture where a central authority oversaw the distribution of energy generated by 
large-scale power plants, predominantly reliant on non-renewable sources like coal and natural gas. Additionally, the large-scale plants producing energy are usually located outside of
cities and far away from consumers. This leads to a big energy waste during the transfer of energy over long distances.
In recent years, ordinary consumers have begun contributing to energy generation through commercially available renewable energy generation units like rooftop solar PV systems. These consumer-producers, also referred 
to as "prosumers" play a crucial role in energy community microgrids. Microgrids are decentralized systems that empower participants to engage in energy trading within local or regional markets. 
They can operate in both grid-connected (meaning they are also connected with the main power grid) or in island-mode (meaning they are completely 
autonomous). An isolated Microgrid has physical connections between the participants that are used to transfer energy and is not connected to the grid. A microgrid connected to the main 
energy grid can also trade energy virtually with other migrogrids or remote participants and the actual energy exchange happens though the DSO.
To facilitate regional microgrid transactions, the concept of a "virtual power plant" (VPP) has been introduced. VPPs aggregate the capacities of various DERs, integrating diverse energy 
sources to ensure a reliable energy supply. \cite{Pipattanasomporn2013,cali2019towards,BrooklynMicrogrid}\\
Microgrids enable two types of energy distribution models, peer to peer (P2P) and peer to grid (P2G).
In P2P model, energy is getting transacted internally in the participants of the microgrid, in a local market.
In P2G model, energy is getting transacted from/to the microgrid to/from the energy grid based on the grid energy market.
To construct a decentralized P2P and/or P2G energy distribution model, it is necessary to ensure the four below propertied for the metered data:
\begin{itemize}
\item \textbf{Accuracy} of the logged energy assets.
\item \textbf{Traceability} of the energy transactions, to ensure the origin of the generated energy and the correct flow of transactions.
\item \textbf{Privacy} of each network participant. It shouldn't be possible to identify the total energy bill of a participant.
\item \textbf{Security} of the energy transactions and the transactions ledger to prevent cyber attacks. \cite{Vangulick2023}
\end{itemize}
Based on the above requirements, blockchain seems to be a viable option for the implementation of a decentralized energy distribution system.\\
Blockchain is an open-source, distributed ledger that can record transactions securely
between two parties, in a verifiable and permanent way. The first running blockchain was implemented by Satoshi Nakamoto in 2008
and it was meant to be a P2P electronic cash system \cite{Nakamoto}. Blockchain technology fullfil the requirements for accuracy, traceability,
and security but lucks on the privacy. Below we analyze each characteristic separately:
\begin{itemize}
    \item \textbf{Accuracy}: Blockchain provides the guaranty that all transactions are immutable and can't be modified once they happen but the accuracy of the data reported to the network is 
    a responsibility of the smart meter devices that measure energy inputs and outputs.
    \item \textbf{Traceability}: All transactions are visible to every participant of the network and can be traced at any time.
    \item \textbf{Privacy}: Due to the fact that all transaction are publicly visible, blockchain doesn't really provide privacy. 
    Despite that, it provides anonymity as all transactions are linked to public keys which doesn't give away the identity of their owner.
    \item \textbf{Security}: All transactions are directly verifiable by all the network participants and they can't be altered. New transactions are added when consensus is reached between the participants. \cite{Vangulick2023}
\end{itemize}

\section{Problem addressed}
\label{problems}
In this research we address the problems that arise from the traditional energy consumption and production model where all the energy is produced in large power plants and transferred over long
distances to households. This approach has the following drawbacks:
\begin{itemize}
    \item \textbf{The consumption of non-renewable sources} like lignite and natural gas that has a negative environmental impact. In addition, they are finite and we can't rely indefinitely on their consumption
    for our energy needs.
    \item \textbf{The energy transfer waste} due to the fact that energy is produces far away from where it is consumed.
    \item \textbf{The energy market censorship} that can arise due to the fact that energy production is centralized and controlled by some central authority.
\end{itemize}
To address these problems we explore decentralized P2P energy trading solutions. P2P energy trading networks rely mostly on renewable energy sources and production of energy happens close to its consumption. In addition, by implementing
the P2P network in a decentralized manner using blockchain technology, we have a trustless architecture where there is no central authority that can censor the energy market. This gives the freedom to network peers to conduct they own
energy trading deals based on their individual needs.

\section{Novelty and contribution}
Our contribution in addressing the problems that arise from traditional energy markets is in advancing the existing decentralized P2P energy trading models by proposing a novel architecture that resolves some of their limitations. 
Most of the implementations we studied rely on a direct energy consumption architecture, meaning that energy is consumed directly when it is produced, with the trading deals happening at an earlier stage. This means that producers need 
to forecast their energy production and trade based on this forecast. The same applies to consumers; they need to forecast the energy to be consumed and make the relevant energy consumption deals with the producers. Forecasting production 
and consumption needs is an error-prone task that can lead to breaches in the energy trading deals. This is especially true when energy production relies on renewable energy sources that can be affected by uncontrollable factors such as weather 
conditions.\\ 
With this study, we introduce an indirect energy trading architecture where energy consumption and production happen asynchronously. This is achieved with the addition of a new role in the energy trading network: the role of energy storage provider, 
which is responsible for storing the energy produced by the producers and providing the stored energy to consumers. In this approach, energy consumption and production can happen at different periods, giving the freedom to network peers to choose when and 
how much energy they want to produce or consume. More information about our proposed architecture can be found in Chapter \ref{chapter4}.

\section{Research structure}
In this study, we structure our research into five basic steps, with each step building upon the previous one. The structure we followed is summarized in the following points:
\begin{enumerate}
    \item Investigation into the fundamentals of blockchain technology and P2P energy trading models (refer to chapter \ref{chapter2}).
    \item Research into the existing literature on P2P energy trading models that make use of blockchain technology (refer to chapter \ref{chapter3}).
    \item Discovery of the limitations of the proposed P2P energy trading models (refer to chapter \ref{chapter3} and chapter \ref{chapter4}).
    \item Proposal of a novel architecture that addresses some of the limitations raised (refer to chapter \ref{chapter4}).
    \item Implementation of a proof of concept energy trading network based on the proposed architecture (refer to chapter \ref{chapter5} and chapter \ref{experimentationANDresults}).
\end{enumerate}
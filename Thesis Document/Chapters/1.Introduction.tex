\chapter{Introduction} \label{introduction}
As the consumer level energy generation solutions become cheaper and more available, new opportunities arise in the energy trading sector. Consumers can produce and trade energy on a 
local network and distribution systems operators (DSO) are utilizing distributed energy resources (DERs) such as rooftop solar photovoltaic units (PV) and wind generators to offer affordable and eco-friendly energy 
solutions.
In the past, electricity distribution primarily adhered to a centralized architecture where a central authority oversaw the distribution of energy generated by 
large-scale power plants, predominantly reliant on non-renewable sources like coal and natural gas. Additionally, the large-scale plants producing energy are usually located outside of
cities and far away from consumers. This leads to a big energy waste during the transfer of energy over long distances.
In recent years, ordinary consumers have begun contributing to energy generation through commercially available renewable energy generation units like rooftop solar PV systems. These consumer-producers, also referred 
to as "prosumers" play a crucial role in energy community microgrids. Microgrids are decentralized systems that empower participants to engage in energy trading within local or regional markets. 
They can operate in both grid-connected (meaning they are also connected with the main power grid) or in island-mode (meaning they are completely 
autonomous). An isolated Microgrid has physical connections between the participants that are used to transfer energy and is not connected to the grid. A microgrid connected to the main 
energy grid can also trade energy virtualy with other migrogrids or remote participants and the actual energy exchange happens though the DSO.
To facilitate regional microgrid transactions, the concept of a "virtual power plant" (VPP) has been introduced. VPPs aggregate the capacities of various DERs, integrating diverse energy 
sources to ensure a reliable energy supply. \cite{Pipattanasomporn2013,cali2019towards,BrooklynMicrogrid}\\
Microgrids enable two types of energy distribution models, peer to peer (P2P) and peer to grid (P2G).
In P2P model, energy is getting transacted internally in the participants of the microgrid, in a local market.
In P2G model, energy is getting transacted from/to the microgrid to/from the energy grid based on the grid energy market.
To construct a decentralized P2P and/or P2G energy distribution model, it is necessary to ensure the four below propertied for the metered data:
\begin{itemize}
\item \textbf{Accuracy} of the logged energy assets.
\item \textbf{Traceability} of the energy transactions, to ensure the origin of the generated energy and the correct flow of transactions.
\item \textbf{Privacy} of each network participant. It shouldn't be possible to identify the total energy bill of a participant.
\item \textbf{Security} of the energy transactions and the transactions ledger to prevent cyber attacks. \cite{Vangulick2023}
\end{itemize}
Based on the above requirements, blockchain seems to be a viable option for the implementation of a decentralized energy distribution system.\\
Blockchain is an open-source, distributed ledger that can record transactions securely
between two parties, in a verifiable and permanent way. The first running blockchain was implemented by Satoshi Nakamoto in 2008
and it was meant to be a P2P electronic cash system \cite{Nakamoto}.
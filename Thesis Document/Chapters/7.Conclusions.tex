\chapter{Conclusions \& Future Work} \label{conclusions}
This research successfully implemented a decentralized energy trading model called PowerChain, utilizing a private blockchain network. 
The results demonstrate the feasibility of our model in facilitating secure, transparent, and traceable energy asset exchange within a local network. 
While PowerChain was proven to be a feasible energy trading model theoretically, more research is needed to practically implement it in a real world 
scenario.\\
Our research was mostly focused on the blockchain layer and the handling of network assets but there is also more investigation needed on the
physical devices that can implement the model in practice. One such device is the smart meter which is an IoT device, controlling a blockchain address, 
responsive to monitor energy production/consumption and report them to the blockchain network. It is a very important part of the network
because it works as an interface between the digital assets and the actual energy production and consumption. More research is needed on how to setup such
devices and how to make sure that the metrics they report can not be tampered with and will always represent the actual metrics with a sufficiently high accuracy.\\
Another aspect that need to be studied further, is the integration of a PowerChain instance with other remote PowerChain instances or even with other public EVM compatible blockchains.
In this study we presented an initial draft of an integration solution in chapter \ref{integration} but more investigation is needed. The integration of a PowerChain instance
with other instances will open the doors to energy trade deals between energy trade communities, allowing to export and import energy to each other. The option to integrate with
public blockchains will allow the bridging of existing assets from the public blockchain into the local PowerChain network. Such assets could be a stable coin that can be used
for the trading of ENT tokens and thus avoid putting trust in a trusted authority to bring EUR assets into the network.\\
A final suggestion for future work, would be to give the possibility to replace the banker role with a community fund. The banker role in the PowerChain network has a lot of power
can could potentially exploit the network. Using the voter roles, it would be possible to implement a community fund that is controlled by many people in the network instead of a
central trusted authority.
